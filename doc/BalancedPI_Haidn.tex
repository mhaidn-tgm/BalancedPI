\documentclass[12p]{article}

\usepackage[english]{babel}
\usepackage[utf8x]{inputenc}
\usepackage[colorinlistoftodos]{todonotes}
\usepackage{fancyhdr} %Package to configure headings and footer
\usepackage{lastpage} %Needed to display last page (total amount of pages)
\usepackage{listings}

% pagelayout
\usepackage[
top    = 2.75cm,
bottom = 2.00cm,
left   = 2.50cm,
right  = 2.00cm]{geometry}
\setcounter{secnumdepth}{4}

% header
\pagestyle{fancy}
\fancyhead[L]{\today}
\fancyhead[R]{Balanced-PI-Calculator - SYT}

%footer
\fancyfoot[L]{Haidn, ???}
\fancyfoot[C]{5A HIT}
\fancyfoot[R]{Seite \thepage/\pageref{LastPage}}

%title page
\author{Martin Haidn, ???}
\title{\Huge{Balanced-PI-Calculator}\\\Large{SYT - 5A HIT}}
\date{\today}

%Glossary
\usepackage{glossaries}
\makeglossaries
	\newglossaryentry{mac} {
		name=MAC,
		description= {Media Access Controll}
	}
	\newglossaryentry{ADC} {
		name= ADC,
		description= {Application Delivery Controller}
	}
	\newglossaryentry{WWW} {
		name= WWW,
		description= {World Wide Web}
	}
	\newglossaryentry{SOA} {
		name= SOA,
		description= {Service Oriented Architecture}
	}
	\newglossaryentry{API} {
		name= API,
		description= {Application Programming Interface}
	}
	
\begin{document}
	\maketitle
	
	\newpage
	\tableofcontents
	
	\newpage
	\section{Aufgabenstellung}
	Es soll ein Load Balancer mit mindestens 2 unterschiedlichen Load-Balancing Methoden (jeweils 7 Punkte) implementiert werden (ähnlich dem PI Beispiel [1]; Lösung zum Teil veraltet [2]). Eine Kombination von mehreren Methoden ist möglich. Die Berechnung bzw. das Service ist frei wählbar!
	
	Folgende Load Balancing Methoden stehen zur Auswahl:
	
	\begin{itemize}
		\item Weighted Round-Round
		\item Least Connection
		\item Least Connected Slow- Start Time
		\item Weighted Least Connection
		\item Agent Based Adaptive Balancing / Server Probes
		\\
	\end{itemize}
	Um die Komplexität zu steigern, soll zusätzlich eine "Session Persistence" (2 Punkte) implementiert werden.\\
	\\
	\textbf{Tests}\\
	Die Tests sollen so aufgebaut sein, dass in der Gruppe jedes Mitglied mehrere Server fahren und ein Gruppenmitglied mehrere Anfragen an den Load Balancer stellen. Für die Abnahme wird empfohlen, dass jeder Server eine Ausgabe mit entsprechenden Informationen ausgibt, damit die Verteilung der Anfragen demonstriert werden kann.\\\\
	\textbf{Modalitäten}\\
	Gruppenarbeit: 2 Personen
	Abgabe: Protokoll mit Designüberlegungen / Umsetzung / Testszenarien, Sourcecode (mit allen notwendigen Bibliotheken), Java-Doc, Jar\\
	\\
	Viel Erfolg!
	
	Quellen
	
	[1] "Praktische Arbeit 2 zur Vorlesung 'Verteilte Systeme' ETH Zürich, SS 2002", Prof.Dr.B.Plattner, übernommen von Prof.Dr.F.Mattern (http://www.tik.ee.ethz.ch/tik/education/lectures/VS/SS02/Praktikum/aufgabe2.pdf)
	[2] http://www.tik.ee.ethz.ch/education/lectures/VS/SS02/Praktikum/loesung2.zip
	
	
	\newpage
	\listoffigures
	\printglossaries
	
	\newpage
	\bibliographystyle{plain}
	\bibliography{Sources.bib}
	
\end{document}